% Set code snippets style
%~ \import{style-code-default}
%~ \import{style-code-bw}
%~ \import{style-code-friendly}
%~ \import{style-code-borland}
%~ \import{style-code-manni}
%~ \import{style-code-trac}
%~ \import{style-code-native}
\import{style-code-lovelace}

\makeatletter
% Control vertical space between Verbatim blocks by using
% the \fanctvrbtopsep and \fancyvrbpartopsep lengths
\newlength{\fancyvrbtopsep}
\newlength{\fancyvrbpartopsep}
\FV@AddToHook{\FV@ListParameterHook}{\topsep=\fancyvrbtopsep\partopsep=\fancyvrbpartopsep}

% Save default value
\let\fancyvrbtopsep@\fancyvrbtopsep
\let\fancyvrbpartopsep@\fancyvrbpartopsep

% Define custom environment CodeSnippet
\DefineVerbatimEnvironment{CodeSnippet@}{Verbatim}
{
numbersep=0pt,
frame=single,
tabsize=1,
fontfamily=DejaVuSansMono-TLF,
fontsize=\relsize{-1},
%~ ,showtabs=true
}
\newenvironment{CodeSnippet}
{%
  \VerbatimEnvironment
  % Remove vertical space
  \setlength{\fancyvrbtopsep}{0pt}
  \setlength{\fancyvrbpartopsep}{0pt}
  \CodeSnippet@%
}
{%Restore default vertical space
  \setlength{\fancyvrbtopsep}{\fancyvrbtopsep@}
  \setlength{\fancyvrbpartopsep}{\fancyvrbpartopsep@}
  \endCodeSnippet@%
}
\makeatother

% Bigger line numbers
\renewcommand{\theFancyVerbLine}{%
  {\ttfamily\normalsize\arabic{FancyVerbLine}}
}
%~ \renewcommand{\FancyVerbTab}{\rule{.1pt}{1ex}}
%~ \renewcommand{\FancyVerbFormatLine}[1]{%
  %~ {\underline{#1\hspace*{\fill}}%
%~ }


% Nicer references to (sub)subsections and appendices
\def\?#1{}
\addto\extrasenglish{
  \def\subsectionautorefname{\?}
  \def\subsubsectionautorefname{\?}
  \def\Appendixautorefname{Appendix}
}

% \nobreakhline : \hline that doesn't break pages
\makeatletter
\def\nobreakhline{%
  \noalign{\ifnum0=`}\fi
    \penalty\@M
    \futurelet\@let@token\LT@@nobreakhline}
\def\LT@@nobreakhline{%
  \ifx\@let@token\hline
    \global\let\@gtempa\@gobble
    \gdef\LT@sep{\penalty\@M\vskip\doublerulesep}
  \else
    \global\let\@gtempa\@empty
    \gdef\LT@sep{\penalty\@M\vskip-\arrayrulewidth}
  \fi
  \ifnum0=`{\fi}%
  \multispan\LT@cols
     \unskip\leaders\hrule\@height\arrayrulewidth\hfill\cr
  \noalign{\LT@sep}%
  \multispan\LT@cols
     \unskip\leaders\hrule\@height\arrayrulewidth\hfill\cr
  \noalign{\penalty\@M}%
  \@gtempa}
\makeatother

% \tablesection{# columns}{text} : used to divide thematic groups in a table
\newcommand{\tablesection}[2]{%
  \multicolumn{#1}{c}%
  {%
    #2%
  }%
  \\* \nobreakhline%
}
% \addcaption{# columns}{text} : reduced caption line
\newcommand{\addcaption}[2]{%
  \multicolumn{#1}{c}{%
    \textbf{[\autoref*{#2}]}%
  }%
}

% Basically semantic \texttt
\CustomVerbatimCommand{\Code}{Verb}{}
\newcommand{\code}[1]{\texttt{#1}}
\newcommand{\variable}[1]{\texttt{\textit{#1}}}
\newcommand{\cmdline}[1]{\texttt{#1}}

% ScheMo deserves a command
\newcommand{\ScheMo}{ScheMo}
%~ \newcommand{\ScheMoTeX}{%
%~ \raise .5ex\hbox {S}%
%~ \kern -.2em\lower .5ex\hbox {\textsc{che}}%
%~ \kern -.12em\raise .1ex\hbox {M}%
%~ \kern -.125em\lower .5ex\hbox {\small O}%
%~ \kern -.125em\TeX{}%
%~ }
\newcommand{\ScheMoTeX}{\ScheMo{}\TeX{}}

% Labels for interface parts
\newcounter{intfcount}    % "Local" counter
\newcounter{intfrefcount} % Reference counter
\renewcommand{\theintfcount}{\Alph{intfcount}}
\renewcommand{\theintfrefcount}{\theintfcount}
\makeatletter
  \newcommand{\intfref}{\@ifstar\@intfref\@@intfref}
  \newcommand{\@intfref}[1]{\ref*{intf:#1}}
  \newcommand{\@@intfref}[1]{\ref{intf:#1}}

  \newcommand{\intflabel}{\@ifstar\@intflabel\@@intflabel}
  \newcommand{\@intflabel}[1]{%
    \stepcounter{intfcount}%
    \refstepcounter{intfrefcount}%
    \label{intf:#1}%
  }
  \newcommand{\@@intflabel}[1]{%
    \@intflabel{#1}%
    \intfref*{#1}%
  }
\makeatother

% Interface environment
\newenvironment{interface}[4][\textwidth]{%
  \setcounter{intfcount}{0}
  \sbox0{\includegraphics[width=#1]{#2}}
  \begin{tikzpicture}[x=\wd0/#3, y=\ht0/#4]
  \node [anchor=north west, inner sep=0] (image) at (-1,1) {\usebox0};
}{%
  \end{tikzpicture}
}

% Questionable typografical choices
\makeatletter
\newcommand{\beforelist}{\@ifstar\@beforelist\@@beforelist}
\newcommand{\@beforelist}{} % Multiple lines before the start of a list
\newcommand{\@@beforelist}{\noindent} % One line before the start of a list

\newcommand{\afterlist}{\@ifstar\@afterlist\@@afterlist}
\newcommand{\@afterlist}{\par} % Multiple lines after the start of a list
\newcommand{\@@afterlist}{\par} % One line before the start of a list

\newcommand{\spacedlist}{\newline \newline}
\makeatother
