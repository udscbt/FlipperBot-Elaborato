In order to create the above networks and share data through them,
all devices must use a common language to communicate.
For this reason, we created the FlipperBot Communication Protocol
(FBCP for short).

\beforelist It is a plain-text, application-level protocol in which each packet
is made of:
\begin{itemize}
  \item a command identifier
  \item zero or more\footnote{The exact number is dependent on the
    command} mandatory parameters separated by spaces
  \item trailing data that can contain optional parameters but is
    often ignored
  \item a newline character (\texttt{0x0A}) to end the packet
\end{itemize}
\afterlist*
\beforelist The various FBCP commands are divided in the following groups:
\begin{itemize}
  \item \textbf{generic}: common commands that can be used with
    other groups of commands
  \item \textbf{network building}: manages requests of permission to
    join a network and grant/denial of it, in addition to changes of
    network topology
  \item \textbf{robot} and \textbf{controller}: commands specific
    for communication with robots and controllers
  \item \textbf{debug}: utilities to thelp in the development of new
    FBCP-compatible software
\end{itemize}
\afterlist*
To make the use of this protocol as easy as possible, we developed
a C++ library and a Python module that implement it.

These tools, however, are only concerned with the creation and
parsing of correct FBCP packets, completely ignoring how the device
should react to them. In fact, a program that uses FBCP need not
implement all supported commands but only the ones it is supposed to
use, giving an error for everything else.

The C++ library and the Python module are also built so to be easily
extendible, making it simple to add new commands to the protocol.

\textit{A full list of currently supported commands with a short
description of each can be found in \autoref{app:fbcp}.}

\subsubsection{Controller modes and options}
  \label{sssec:optmode}
  Given the wide range of possible controllers that could be used
  with this game (the ones described in this text being only a part
  of them), it can be useful to also have different ways of managing
  the sharing of information between controller and server (whether
  it is a robot or a board).

  To this end, it is possible to set or unset certain
  \textit{options} for the connection, flags that indicate if
  certain modifications must be applied to the normal way of
  communicating. As an example, for the accelerometer-based
  controller described in \autoref{ssec:accel}, an useful option could
  be one that removes the need for the controller to send a command
  to make the robot go forwards, having the server assume that that
  is the desired direction when no command is sent.

  Other than being set or unset, options can, if needed, take a
  particular value, making them more generic and versatile. There
  could be, for example, an option called \texttt{timeout} that
  contains information about how long the server should wait for a
  command before considering the controller disconnected.

  In the case that there are many options to be set or unset for the
  correct use of a controller, it is easier to use \textit{modes}. A
  mode is nothing more than a predefined collection of options, with
  their set/unset states (and their additional values if needed)
  already decided.

  \beforelist* Enabling a mode is in no way different from setting the individual
  options one by one from a functionality point of view, but:
  \begin{itemize}
  \item it is quicker
  \item it requires fewer package exchanges
  \item it makes it possible to modify the options needed by the
    server to make a particular controller work without editing the
    code of the controller (if the name of the mode doesn't change)
  \end{itemize}
  \afterlist*
  \beforelist* A server isn't required to support all possible modes and options,
  so a controller should:
  \begin{enumerate}
  \item try to select all suitable modes, from the one that allows
    the best performance to the worst
  \item try to set the necessary options one by one, if the previous
    point failed
  \item fall back to the standard way of communicating (with direct
    robot or motor commands) if the available options aren't
    sufficient
  \end{enumerate}
  \afterlist*
