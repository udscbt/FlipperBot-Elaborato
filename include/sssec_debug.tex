This package was created to facilitate the detection of errors in the code
during development and to test the functionality of completed parts.

\beforelist It's main features are:
\begin{itemize}
  \item managing the output to screen of informative messages and errors, with
    the automatic addition of useful information such as the module and class
    that generated the message;
  \item writing logs with timestamps and tags that can be used to filter the
    desired data from the rest;
  \item visualising a summary of all important data of the game.
\end{itemize}
\afterlist*
The last point is accomplished through the \Code|debug.monitor.Monitor| class,
that generates a text-based interface that can be either printed to screen only
when desired, or configured to continously display the state of the game and
automatically update the information after regular intervals.

This interface can be seen in \autoref{fig:monitor}.

\begin{figure}[h]
  \import{fig_monitor}

  \caption[\code{Monitor} interface]{\Code|Monitor| interface}
  \label{fig:monitor}
\end{figure}

Although this interface could in theory be used to play the game (at least in
Simon Says mode) without a real board at disposal, a better option would be to
use the \Code|demo| package (described below).
