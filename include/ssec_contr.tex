\beforelist* The ESP-12 on the joystick is programmed in C++, using the
\ScheMo{} library to allow mutiple threads to run simultaneously.
The joystick must perform three basic activities at the same time, assosciated
to the same number of jobs:
\begin{description}
  \item[\Code|job\_client|:] gathers data from the hardware (the button and the
    two potentiometers) and if:
    \begin{enumerate}[itemjoin={,}, itemjoin*={{ and }}]
      \item the joystick is connected to another device and
      \item the information has changed from the last transmission
    \end{enumerate}
    it sends the suitable FBCP commands to the server;
  \item[\Code|job\_link|:] manages the blue LED, turning it on and off at the
    appropriate frequency depending on the current state;
  \item[\Code|job\_network|:] searches for available board or robot networks,
    tries to connect to them, and puts the joystick in the right state
    depending on the result, repeating this operation each time it finds itself
    disconnected.
\end{description}
\afterlist*
Since the preferred way to use a controller is through a game network
(i.e.\ connected to a board), the scanned SSIDs are sorted in alphabetical
order before trying to connect to them one at a time.
This way, connection to the \code{FlipperBot-Board-\ldots} networks will always
be attempted before the \code{FlipperBot-Robot-\ldots} ones.

While there are specific commands to end a connection, it is possible that
a devices breaks the connection without explicitly stating so.
This is the case, for example, when a device is turned off (without a source of
power it is impossible for it to send the right message).

To prevent problems in such cases, every once in a while the joystick will send
a \emph{keepalive} command\footnote{Called \Code|Q\_HEARTBEAT| in
\autoref{app:fbcp}.} to the server to check if the device is still actually
connected.
If no response is received, the joysticks falls back to the idle state and tries
to connect to a new network.
