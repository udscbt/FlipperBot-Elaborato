The Arduino controlling the display accepts string messages and parses them to
understand what text to show on the display and how to do it.

A message is a list of zero or more colon-separated symbols, terminated by a
semicolon.
A symbol represents what should be shown in one of the four seven-segments units
of the display and it can be selected from \autoref{table:dispsym}.
Each one of the default symbols also have a \emph{dotted} version, enabled by
adding a dot after the symbol\footnote{E.g.\ ``\Verb+0+''$\rightarrow$
``\Verb+0.+''.}, that lights up the decimal-point dot of the seven-segments unit
in addition to the segments that form the symbol.
Additionally, symbols can be combined together by placing a \Verb|+| between
them\footnote{So that ``\Verb|r+J|'' has the same meaning of ``\Verb|d|''.}.

\beforelist* In any point of the message, an \emph{options} command can be
inserted.
It is delimited by parentheses and contains a list of zero or more
semicolon-separated options.
In the current version of the program, up to three options can be interpreted
(the other ones being ignored if present) and they represent:
\begin{enumerate}
  \item the refresh rate
  \item the scroll speed
  \item the blink frequency
\end{enumerate}
\afterlist*
An option can be either be an empty string (in which case the old value is kept)
or a numerical value, optionally with a prefix specifying the numerical base
used.
If the chosen value is $0$, the corresponding feature is disabled.
All options are applied as soon as the closing parenthesis is encountered.

\emph{Note: if a base is specified but no number is inserted, the value of the
option is treated as 0.}

A formal definition in Extended Backus-Naur form\cite{ebnf} of this structure
can be seen in \autoref{fig:dispmsg}.

\begin{figure}[h]
\import{display-ebnf.tex}

\caption{Formal grammar definition of display messages}
\label{fig:dispmsg}
\end{figure}

Actually, the program tries to parse any message it receives, even if it isn't
correctly formatted; thus, it is possible to obtain the desired result using a
slightly different syntax than the one presented, but the behaviour is clearly
defined only for correctly formatted messages.

\begin{table}[h]
\begin{multicols}{6}
\makeatletter\col@number\@ne\makeatother
\begin{longtable}{l|c}
  \import{tab_dispsymb}
\end{longtable}
\end{multicols}

\caption{List of displayable symbols}
\label{table:dispsym}
\end{table}
