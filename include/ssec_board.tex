The Raspberry Pi on the board runs the Raspian Wheezy operating system, a GNU/Linux
system built specifically for this device.

All the software we developed to run on the RPi was written in Python,
specifically it was tested on version \Verb|3.5.2|.

\beforelist* Considering the large number of different, and sometimes unrelated,
tasks the board has to carry out, we opted for a very modularized structure.
The \Code|flipperbot.board| package, containing all the Python code specifically
developed for the board, is in fact divided in the following modules:
\begin{multicols}{2}
\begin{itemize}
  \item \Code|audio|
  \item \Code|display|
  \item \Code|everythingButton|
  \item \Code|game|
  \item \Code|led|
  \item \Code|server|
  \item \Code|shared|
  \item \Code|simonsays|
  \item \Code|totems|
\end{itemize}
\end{multicols}
\afterlist*
Each of these has a specific function and they have been designed in such a way
that their implementation can be radically changed, while their interface to the
other modules remains the same.
As an example of this, when we switched from an internal to an external display controller, the \Code|display| module has been almost completely rewritten, but
not a single line of the other modules needed edits.
This means that future improvements of the game will be possible without losing
any of the previous work.

Among these modules, the most important is \Code|game|.
Its role is to use all other modules to accomplish the common goal of making the
game work as wanted.
While all other modules\footnote{Except for \Code|simonsays|, which contains alternative game rules} concern themselves with managing the interactions with
the external world, \Code|game| takes the data retrieved from the former and
gives them meaning by implementing the rules of the game, allowing the inputs
and outputs to cooperate in a coordinated way.

In addition to these modules, the \Code|board| package contains two subpackages
not directly related to making the game work: \Code|debug| and \Code|demo|. 

\subsubsection[\code{demo} package]{\Code|debug| package}
\import{sssec_debug}

\subsubsection[\code{demo} package]{\Code|demo| package}
\import{sssec_demo}

\subsubsection{External display}
\import{sssec_display}
