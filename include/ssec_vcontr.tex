As explained in \autoref{ssec:vint}, we developed a virtual interface that can
be used instead of the real joystick.

This application is written in Python and uses the Tkinter package for the
graphical interface; it can thus be run on a wide range of devices, from PCs
to Android smartphones (although not natively for the latter).

Note that this program is not to be intended as a definitive and unimprovable
alternative version to the standard controller, but more as a proof of concept.
In fact, it shows that the way the transission of commands is handled allows for
different kind of controllers, making it possible to use or build the most
suitable for the individual user.

This virtual interface fully takes advantage of the \Verb|flipperbot.fbcp|
module to implement the protocol in a simple and efficient way and the same
could be done when developing in C++ instead of Python. While no other languages
are currently supported\footnote{While the lack of a Java library seems to
indicate the current impossibility to write an Android app, it is theoretically
possible to use the C++ library in conjunction with the Android NDK.}, the simple structure of FBCP makes it easy
to implement it in a language of choice without any deep technical knowledge.
