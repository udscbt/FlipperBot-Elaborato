\autoref{table:fbcp_cmd} lists all currently supported FBCP commands.

The first column contains the name with which each command is referred
to in code (both using the C++ library and the Python module).
The first letter indicates wether the command should be sent from the
client (\textbf{Q}uestion) or from the server in response to a previous
command (\textbf{A}nswer).

The second one shows how a packet must be written to be understood as
the related command (the words in italic are mandatory parameters and
should be replaced with appropriate strings). The terminating
\textit{newline} character has been omitted for the sake of readability.

The last column briefly explains how each command is used.

The various commands are divided in the groups described in
\autoref{ssec:fbcp}.

\def \cmddescwidth {0.3\textwidth}
\begin{longtable}{llp{\cmddescwidth}}
\addcaption{3}{table:fbcp_cmd}
\\
\textbf{Command name}
& \textbf{Packet structure}
& \textbf{Description}
\\ \hline

\hline
\endhead

\addcaption{3}{table:fbcp_cmd}
\endfoot

\endlastfoot

\tablesection{3}{\textit{Generic}}

\code{A\_ACCEPT}
& \code{ON\_IT!}
& Sent if the previous command was accepted and correctly handled.
\\ \hline
\code{A\_REFUSE}
& \code{NOT\_NOW}
& Sent if the previous command was not handled as requested.
\\ \hline
\code{A\_ERROR}
& \code{WHAT?}
& Sent if the previous command couldn't be understood or wasn't a
correct FBCP packet.
\\ \hline \hline

\tablesection{3}{\textit{Network building}}

\code{Q\_SINGLE\_PRESENTATION}
& \code{I'M \variable{serial}}
& Asks for permission to join a network. \variable{serial} is an unique
identifier of the device\footnote{It must be in the form
\code{FlipperBot-Board-\ldots} for boards,
\code{FlipperBot-Robot-\ldots} for robots and
\code{FlipperBot-Controller-\ldots} for controllers}.
\\ \hline
\code{Q\_MULTI\_PRESENTATION}
& \code{WE'RE \variable{serial} \variable{friend}}
& Asks for permission of the requesting device and another one to join
a network. Used when already connected robots and controllers want to
join a game network. The \variable{friend} device will have a reserved
place in the network but will stillr need to send a
\code{Q\_SINGLE\_PRESENTATION} to the server.
\\ \hline
\code{A\_GRANT\_ACCESS}
& \code{WELCOME}
& Sent if a request to join a network is accepted.
\\ \hline
\code{A\_DENY\_ACCESS}
& \code{BUSY}
& Sent if a request to join a network is refused. Usually because there
are already enough devices connected.
\\ \hline
\code{Q\_HEARTBEAT}
& \code{STILL\_THERE?}
& \multirow{2}{\cmddescwidth}{
Used to avoid timeouts.
}
\\ \cline{1-2}
\code{A\_HEARTBEAT}
& \code{YEP}
%& see above
\\ \hline
\code{Q\_CLEAN}
& \code{GOODBYE}
& \multirow{2}{\cmddescwidth}{
Used to disconnect a device from the network.
}
\\ \cline{1-2}
\code{A\_CLEAN}
& \code{BYE}
%& see above
\\ \hline
\code{A\_REQUEST\_PRESENTATION}
& \code{WHO?}
& Sent if the previous command was sent without a prior proper
connection.
\\ \hline
\code{Q\_CHANGE\_NET}
& \code{MOVE\_TO \variable{net}}
& Used by a robot to inform a connected controller of the wish to change
network. The robot must have already sent a
\code{Q\_MULTI\_PRESENTATION} command to the new network to reserve a
place in the new network for the controller.
The \variable{net} parameter must be set to the SSID of the new network.
\\ \hline
\code{A\_CHANGE\_ACCEPT}
& \code{OK}
& Sent by a controller to accept a change of network. It must then send
a \code{Q\_SINGLE\_PRESENTATION} to the new network to effectively
join it.
\\ \hline
\code{A\_CHANGE\_DENY}
& \code{NO}
& Sent by a controller if it isn't willing to join the new network. In
answer to this, the robot must leave the new network (that was joined in
combination with the controller) and then either:
\begin{itemize}
\item leave the old network too and rejoin the new one alone
\item remain in the old network
\end{itemize}
\rule{0pt}{0pt}
\\ \hline \hline

\tablesection{3}{\textit{Robot}}

\code{Q\_MOTOR\_COMMAND}%
\footnote{\label{motcmd_note}These commands are also part of the
  \textit{Controller} group}
& \code{MOTOR \variable{motor} \variable{direction}}
& Sets the direction of rotation of a specific motor%
  \footnote{\label{param_note}See \autoref{table:fbcp_param} to
    find available parameters}.
\\ \hline
\code{Q\_ROBOT\_COMMAND}\footref{motcmd_note}
& \code{BOT \variable{direction}}
& Sets the direction of motion of the whole robot%
  \footref{param_note} by changing the speed of both
  motors (not necessarily in the same way).
\\ \hline
\code{Q\_HIT}
& \code{OUCH}
& Sent when the bumper of the robot hits something.
\\ \hline \hline

\tablesection{3}{\textit{Controller}}

\code{Q\_OPTION}
& \code{OPT \variable{option} \variable{value}}
& Sets/unsets (based on \variable{value}\footref{param_note}) an option%
\footnotetext{\label{optmode_note}See \autoref{sssec:optmode} for an
explanation of options and modes}. For certain options, the value
parameter can contain additional information.
\\ \hline
\code{A\_OPTION\_ACCEPT}
& \code{OK}
& Sent if the requested option was recognized and correctly set/unset.
\\ \hline
\code{A\_OPTION\_DENY}
& \code{DUNNO}
& Sent if the requested option wasn't recognized or its state couldn't
be changed as wanted.
\\ \hline
\code{Q\_MODE\_SELECT}
& \code{MODE \variable{mode}}
& Enables a certain mode\footref{optmode_note}.
\\ \hline
\code{A\_MODE\_ACCEPT}
& \code{OK}
& Sent if the requested mode was recognised and enabled.
\\ \hline
\code{A\_MODE\_DENY}
& \code{DUNNO}
& Sent if the requested mode wasn't recognised or couldn't be enabled.
\\ \hline
\code{Q\_EVERYTHING\_ON}
& \code{PRESSED}
& Asks to set all available options.
\\ \hline
\code{Q\_EVERYTHING\_OFF}
& \code{RELEASED}
& Asks to unset all available options.
\\ \hline
\code{Q\_RAW\_COMMAND}
& \code{RAW}
& Used to send any kind of data with the assumption that the server will
correctly handle it and use it to guide the robot. The right
mode/options should be selected in advance.
\\ \hline \hline

\tablesection{3}{\textit{Debug}}

\code{A\_DATA}
& \code{DATA}
& Can be used in various contexts and there isn't a specific way of
handling it. It can contain any kind of data as optional parameter with
the assumption that the requesting device knows how to interpret it.
\\ \hline
\code{A\_ALIKE}
& \code{MAYBE \variable{command}}
& Suggests a command similar with a similar signature to the received
one in case the latter is not recognized. It makes recognizing typos
easier.
\\ \hline
\code{Q\_LIST}
& \code{LIST \variable{type}}
& Requests a list of supported commands or options, depending on the
\variable{type} parameter\textsuperscript{\ref{param_note}}. The answer
should be a \code{A\_DATA} command with a list of comma-separated
commands/options as optional parameter.
\\ \hline
\code{Q\_HELP}
& \code{EXPLAIN \variable{command}}
& Requests a description of how to use the command and/or what it does.
The answer should be a \code{A\_DATA} command.
The implementation can give a more or less detailed answer that isn't
usually suitable for automatic parsing but is more useful for
interactive interfaces during debugging.
\\ \hline
\code{Q\_LOG}
& \code{LOG}
& Requests the addition of some data to the server logs.
\\
\caption{List of supported FBCP commands}
\label{table:fbcp_cmd}
\end{longtable}

\autoref{table:fbcp_param} contains a list of all predefined parameter
strings to be used in conjunction with the commands in \autoref{table:fbcp_cmd}.

The first column contains the name with which each parameter is referred
to in code (both using the C++ library and the Python module).

The second one shows what the corresponding string is.

The last column explains the meaning of the parameter.

The various entries are divided based on the commands with which they
are supposed to be used; the mandatory parameter to which the string
refers is indicated in parenthesis.

\def \pardescwidth {0.4\textwidth}
\begin{longtable}{llp{\pardescwidth}}
\addcaption{3}{table:fbcp_param}
\\
\textbf{Parameter name} & \textbf{String} & \textbf{Description} \\
\hline

\hline
\endhead

\addcaption{3}{table:fbcp_param}
\endfoot

\endlastfoot

\tablesection{3}{\code{Q\_OPTION} (\variable{value})}

\code{OPTION\_SET}
& \code{ON}
& \multirow{2}{\pardescwidth}{
  Indicate whether the related \textit{option} must be set or unset.
}
\\ \cline{1-2}
\code{OPTION\_UNSET}
& \code{OFF}
%& see above
\\ \hline \hline

\tablesection{3}{\code{Q\_LIST} (\variable{type})}

\code{LIST\_OPT}
& \code{OPT}
& Used to list options.
\\ \hline
\code{LIST\_CMD}
& \code{CMD}
& Used to list commands.
\\ \hline \hline

\tablesection{3}{\code{Q\_MOTOR\_COMMAND} (\variable{motor})}

\code{MOTOR\_LEFT}
& \code{ML}
& Indicates that the left motor speed must be changed.
\\ \hline
\code{MOTOR\_RIGHT}
& \code{MR}
& Indicates that the right motor speed must be changed.
\\ \hline
\code{MOTOR\_BOTH}
& \code{MLR}
& Indicates that both motors speed must be changed to the same value.
\\ \hline \hline

\tablesection{3}{\code{Q\_MOTOR\_COMMAND} (\variable{direction})
  and \code{Q\_ROBOT\_COMMAND} (\variable{direction})}

\code{DIRECTION\_FORWARD}
& \code{FW}
& Changes the speed of the motor/motors to move the robot forwards.
\\ \hline
\code{DIRECTION\_BACKWARD}
& \code{BW}
& Changes the speed of the motor/motors to move the robot backwards.
\\ \hline
\code{DIRECTION\_STOP}
& \code{STOP}
& Stops the motor/motors.

\\ \hline \hline

\tablesection{3}{\code{Q\_ROBOT\_COMMAND} (\variable{direction})}

\code{DIRECTION\_FORWARD\_LEFT}
& \code{FL}
& Changes the motors speed to make the robot turn left while moving
  forwards.
\\ \hline
\code{DIRECTION\_FORWARD\_RIGHT}
& \code{FR}
& Changes the motors speed to make the robot turn right while moving
  forwards.
\\ \hline
\code{DIRECTION\_BACKWARD\_LEFT}
& \code{BL}
& Changes the motors speed to make the robot turn left while moving
  backwards.
\\ \hline
\code{DIRECTION\_BACKWARD\_RIGHT}
& \code{BR}
& Changes the motors speed to make the robot turn right while moving
  backwards.
\\ \hline
\code{DIRECTION\_LEFT}
& \code{SL}
& Changes the motors speed to make the robot turn left in place.
\\ \hline
\code{DIRECTION\_RIGHT}
& \code{SR}
& Changes the motors speed to make the robot turn right in place.
\\
\caption{List of standard FBCP parameters}
\label{table:fbcp_param}
\end{longtable}
