As stated in \autoref{ssec:display}, the display is not directly controlled by
the Raspberry Pi on the board but by a dedicated Arduino.

The program loaded on the Arduino is written in C++ and receives commands%
\footnote{More details about the structure of these commands can be found in
\autoref{app:display}.} from an external source (in this case from the Raspberry
Pi) through a TTL serial connection with baudrate of 9600 bps.

\beforelist By sending the right string to the Arduino, it is possible to:
\begin{itemize}
\item set the displayed text (if it's longer than 4 character, it will be
  shown by making it scroll through the display)
\item set the refresh rate (i.e.\ the frequency in Hz at which the characters
  are ``redrawn'')\footnote{While higher refresh rates are usually preferred to
  obtain the illusion of a static text, lower ones can be sometimes used for
  interesting effects.}
\item set the scroll speed (in characters per second)
\item enable/disable blinking
\item set the blinking speed
\end{itemize}
\afterlist*
The program is capable to translate various characters into the right set of
segments in the display, but if a specific character isn't available it can
still be displayed as a combination of the default ones (in the worst case by
setting the individual segments).
