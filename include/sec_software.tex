In addition to designing and building the physical parts of this
project, a considerable amount of work was put in developing the
software needed to make the various devices operate as wanted.

The main programming languages used were C++ and Python, with some
utilities written as Bash scripts.

\autoref{fig:softtree} shows the main software components of this project,
ignoring the utility tools only used during development.

Snippets of the code contained in this project can be found in
\autoref{app:code}, presented as examples of use of the various tools.

\begin{figure}[htbp]
  \import{fig_softtree}
  \caption{Structure of the main software components of this project}
  \label{fig:softtree}
\end{figure}

\subsection{Scheduler Module (\ScheMo{})}
  \label{ssec:schemo}
  \import{ssec_schemo}

\subsection{Network topology}
  \label{ssec:network}
  \import{ssec_network}

\subsection{FlipperBot Communication Protocol (FBCP)}
  \label{ssec:fbcp}
  \import{ssec_fbcp}
  
\subsection{Joystick}
  \label{ssec:contr}
  \import{ssec_contr}

\subsection{Robot}
  \label{ssec:robot}
  \import{ssec_robot}

\subsection{Board}
  \label{ssec:board}
  \import{ssec_board}
  
\subsection{Virtual Interface}
  \label{ssec:vcontr}
  \import{ssec_vcontr}
