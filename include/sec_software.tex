In addition to designing and building the physical parts of this
project, a considerable amount of work was put in developing the
software needed to make the various devices operate as wanted.

The main programming languages used were C++ and Python, with some
utilities written as Bash scripts.

Snippets of the code contained in this project can be found in
\autoref{app:code}, together with examples of use of the various tools.

%~ \begin{figure}[htbp]
  %~ \import{fig_softtree}
  %~ \caption{Structure of the libraries and modules developed for this project}
  %~ \label{fig:softtree}
%~ \end{figure}

\subsection{Scheduler Module (\ScheMo{})}
  \label{ssec:schemo}
  \import{ssec_schemo}

\subsection{Network topology}
  \label{ssec:network}
  \import{ssec_network}

\subsection{FlipperBot Communication Protocol (FBCP)}
  \label{ssec:fbcp}
  \import{ssec_fbcp}
  
\subsection{Robot}
  \label{ssec:robot}
  \import{ssec_robot}
  
\subsection{Controller}
  \label{ssec:contr}
  \import{ssec_contr}
  
\subsection{Board}
  \label{ssec:board}
  \import{ssec_board}
  \subsubsection{\code{demo} module}
  
\subsection{Virtual controller}
  \label{ssec:vcontr}
  \import{ssec_vcontr}

\subsection{Utilities}
  \label{ssec:utils}
  \import{ssec_utils}
  \subsubsection{Projects management}
  \subsubsection{Makefile}
