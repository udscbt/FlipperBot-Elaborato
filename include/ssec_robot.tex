The robot's software is similar to that of the joystick in many ways, so only
the main differences are listed here.

\beforelist The jobs executed by the robot are:
\begin{description}
  \item[\Code|job\_client|:] it is identical to that of the joystick except that
    it takes data from the bumper (and thus the sent command is different);
  \item[\Code|job\_link\_and\_bumper|:] it manages two sets of LEDs instead of
    only one like the joystick and, in addition to this, it uses information
    from the bumper to change the current direction\footnote{These two unrelated
    functions have been united in the same job only because they are so simple
    and short that splitting them in two jobs would only be a waste of
    resources.};
  \item[\Code|job\_network|:] similar to the one in joystick, but only tries to
    connect to boards, getting into standalone mode if it fails;
  \item[\Code|job\_server|:] each time the robot changes state, it accepts
    exactly one connection from a device (the board to which the robot is
    already connected as a client when in game mode, or a new controller when in
    standalone mode) and executes the received commands related to its direction
    or the motors speed.
\end{description}
\afterlist*
Just like the joystick, the robot sends a keepalive signal when necessary to
check whether the connection is still active.
It is to note, however, that the client job is active only while in game mode
(in fact, the joystick wouldn't handle the bumper information even if it
received it since it doesn't run a server).

However, unlike the joystick the robot also has a server that needs to assure
the functionality of the connection.
This is accomplished by setting a timeout (longer than the normal time between
a keepalive command and the next) after which the connection is broken if no
command is received.
Of course, it also replies to incoming keepalive commands.

As said in the list above, the bumper is used to change the direction of the robot
(in addition to sending its raw value to the board).
In fact, to avoid having the robot needlessly pushing against the walls, the
forward direction is prevented when the bumper is pressed (the robot stops
instead).
Additionally, the forward left and forward right direction are converted
to in-place left and right rotations respectively.
