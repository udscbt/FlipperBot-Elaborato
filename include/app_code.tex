\section{C++}
Following is presented a reduced version of the program loaded on the robot,
used as an example both to explain some of the features of \ScheMo{} and FBCP,
and to show how those features can be applied in practice.

This code is not compilable by itself, but shows significant parts of the
original program, all extensively commented.

The functions related to the Arduino IDE and the \Verb|ESP8266WiFi| library are
mostly left uncommented, since an explanation of those tools is out of the
purpose of this document.

\import{robot-cpp}

Moreover, to report an example of \ScheMoTeX{}, \autoref{fig:code_example} shows the
generated diagrams of some components of this same program. In particular,
the first function is also (partly) present in the above code.

Other components of the program haven't been inserted because they couldn't fit
in pages without becoming illegible.

\begin{figure}[hp]
  \schemoSetName{fig_code_example}
  \import{fig_code_example}
  \caption{Partial flow diagram of ESPer's program}
  \label{fig:code_example}
\end{figure}

\section{Python}
Below there are some excerpts from the \Verb|flipperbot.board.server| module
used in the board. They show the main features of the \Verb|flipperbot.fbcp|
module, the Python version of the FBCP C++ library.

Since most of these features have already been explained in the C++ example,
the conversion of data structures and functions from C++ to Python is indicated
by comments in the form \Verb|# Cpp_version -> Python_version| placed above the
related piece of code, with only significant differences explicitly stated.

\import{server-py}
