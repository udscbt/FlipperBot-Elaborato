\begin{figure}[h]
  \import{fig_vcontr}

  \caption[Virtual joystick interface]{
    \centering
    Virtual joystick interface

    \small\noindent
    \begin{enumerate*}[itemjoin={{; }}, after=.]
    \item[\bfseries \intfref{connect}:] Connection button
    \item[\bfseries \intfref{joystick}:] Virtual joystick
    \item[\bfseries \intfref{options}:] Options
    \item[\bfseries \intfref{hint}:] Hint text
    \end{enumerate*}
  }
  \label{fig:vcontr}
\end{figure}

\beforelist \autoref{fig:vcontr} shows a screenshot of the interface. The labels
refer to its components:
\begin{description}
\item[Connection button:] allows to start and stop a connection with a board
  or a robot (the WiFi connection to the right network must be done outside of
  this program);
\item[Virtual joystick:] simulates a real joystick, showing its current
  position and accepting user input;
\item[Options:] allow to enable or disable certain features of the virtual
  interface on the fly, enhancing the user experience;
\item[Hint text:] briefly explains what is the meaning of an option when the
  cursor goes over it.
\end{description}
\afterlist*
Depending on the selected options, it is possible to control the device either
with a keyboard (arrow keys or number pad) or with the mouse (or with both at
the same time). Regardless of the enabled input method for controlling the
joystick, a virtual everything button is always bound to the \Verb|Return|
keyboard key.

If the child commonly uses a hardware or software solution
that simulates a keyboard or a mouse, it can be used with this interface too.
