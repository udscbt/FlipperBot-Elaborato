FlipperBot is made of different devices that need to communicate
among themselves and for this reason they have to be connected in a
suitable network. In particular, the devices can be connected in
different ways depending on the current mode: standard game or
standalone.

\subsubsection{Standard Game Network (SSID starting with
  \code{FlipperBot-Board-})}
  \begin{figure}
    \import{fig_netgame}
  \end{figure}
  In game mode, the board acts as access point for all other devices
  and all communications go through it.

  In particular, in this mode controller and robot are not directly
  connected, instead there is a client-server connection between
  controller and board and another one between board and robot. This
  way, the board is able to block or modify the commands given by
  the controller before sending them to the robot, as suited to make
  the game work as expected.

  The first connection is also used to manage the Everything Button
  on the controller.

  Another client-server connection exists between the robot and the
  board to allow the transmission of impact information.

\subsubsection{Standalone Network (SSID starting with
  \code{FlipperBot-Robot-})}
  When no boards are available for connection, a robot can become an
  access point to create a new network.

  A single controller can join this network and create a
  client-server connection with the robot, directly controlling it.

\subsubsection{Isolated Network}
  In the case in which neither boards nor robots are available, the
  controller simply doesn't connect to any other device while
  waiting for a suitable network to appear.

\subsection*{}
The board always uses the game mode, broadcasting the network SSID%
\footnote{Service Set IDentifier: the \textit{name} of the WiFi
network} to make its presence known. Robots try to connect to a game
network if they can find it, otherwise they get into standalone
mode. Controllers connect to whichever suitable network they can
find, giving priority to the ones created by a board.

In some cases, a WiFi network can exist and be accessible while it
is already \textit{full} (all devices of the previously defined
topologies are already present). Such cases are managed by FBCP%
\footnote{See \ref{ssec:fbcp}} and lead to changes in topologies.

