Both the robot and the controller (as will be clear later) need to
execute multiple tasks simultaneously to work properly, however the
ESP8266 includes only a one-core CPU, allowing for only one process
to run at a time.

The SDK\footnote{Software Development Kit} available to be used with
the Arduino IDE\footnote{Available at
\textit{https://github.com/esp8266/Arduino}} implements a simple
scheduler to allow the correct functioning of the WiFi communication
while the main program runs. Unfortunately, the related API are hard
to use, badly documented and their use is discouraged by the
developers.

For these reasons, we developed our own scheduler, written in
standard C++ and thus compatible with various platforms other than
the ESP8266. It is, however, only a barebone implementation useful
for this specific application but lacking of various features common
to modern schedulers.

\subsubsection{Control flow}
  \begin{figure}[htbp]
    \schemoSetName{fig_schemoflow}
    \centerline{
      \import{fig_schemoflow}
    }
    \caption[Example flow diagram of a \ScheMo{} program.]{
      Example flow diagram of a \ScheMo{} program.

      \small
      Each job has a distinct flow diagram, where tasks are represented
      by circles.

      The padlocks indicate if a mutex is locked/unlocked inside a task
      and they delimit the critical sections of the program (the same
      kind of hatch means that both critical sections refer to the same
      mutex).

      The small numbers near the tasks show the order with which they
      are executed, alternating between jobs and remaining stuck at the
      start of critical sections when the related mutex is busy.
    }
    \label{fig:schemoflow}
  \end{figure}

  \beforelist* To better understand how \ScheMo{} simulates parallel processing,
  we first have to define some of its terminology%
  \footnote{Note that these definitions are specific for \ScheMo{} and
  do not necessarily correspond to the meanings that the same words
  have in other contexts of Computer Science}:
  \begin{description}[align=right]
    \item [Job] it represents a single (pseudo-)process that can be
      started, stopped, paused and resumed at will. Different
      processes are represented by different jobs.
    \item [Task] it's a continous, non-interruptable part of a job;
      that is, the scheduler can switch control from a job to
      another only after a task of the first job has terminated and
      before the next task starts. This means that it is vital to
      split jobs in as many tasks as possible and to make sure that
      tasks always halt (this condition is not required for jobs)
    \item [Function] it's similar to a job in that it can run
      concurrently with other processes, but it has some important
      differences:
      \begin{itemize}
        \item it can't be scheduled by itself, it must instead be
          called inside a job (that passes control to the function
          until completion\footnote{Asynchronous function calling
          has not been implemented yet})
        \item when called, it can accept parameters that change its
          behaviour
        \item once finished, it returns a value that can be used for
          subsequent computation\footnote{\code{void} functions are
          not implemented as of now}
        \item multiple instances of the same function can be run at
          the same time without conflicting with each other
      \end{itemize}
    \item [Cycle] it represents the order of execution of the
      various tasks. The name comes from the fact that, at any one
      time, it only contains a small portion of the tasks necessary
      for the program and its content is updated by a loop inside
      the core function of the scheduler.
  \end{description}
  \afterlist*
  \ScheMo{} manages execution of tasks in a very simple way: it has a
  loop that, at each iteration, cycles through all started jobs and
  adds the current task of each to the Cycle (potentially skipping
  some jobs if they are locked\footnote{See below} or defined to be
  run less often) and then proceeds to execute them. Once all jobs
  have terminated, the scheduler stops\footnote{It is therefore
  important to schedule at least one job before starting the
  scheduler}.

  A new job can be started by informing the scheduler of its first
  task and asking for it to be executed.

  Each task, then, must tell the scheduler what the next task in the
  job is before terminating (this process can be automated by
  \cmdline{schemop}\footnote{See \autoref{sssec:schemop}}).

  \ScheMo{} also incorporates a system to manage shared memory between
  jobs by using mutexes.

  \beforelist* A mutex is a dummy variable associated to a block of memory that
  keeps track of whether or not that block of memory is being used by
  some job. A job that wants to access that data must make a request
  to the related mutex, with two possible outcomes:
  \begin{enumerate}
    \item the mutex changes state from \textit{free} to
      \textit{busy} and the job continues its execution
    \item the mutex remains in the \textit{busy} state and the job
      locks until the mutex is freed
  \end{enumerate}
  \afterlist*
  To allow other jobs to use the same resources, a job must free the
  mutex once it has finished accessing the data.

  See \autoref{fig:schemoflow} to see a graphical representation
  of how \ScheMo{} operates.

  \textit{Note: \ScheMo{}'s management of shared resources is very
  basic and it is not able to prevent deadlocks or other similar
  problems related to mutexes.}
 
\subsubsection{\ScheMo{} Preprocessor (\cmdline{schemop})}
\label{sssec:schemop}
  Even if the C++ library offers various functions and macros to
  ease the writing of a multithreaded program, complex applications
  still result in hard-to-read and cluttered code.

  To avoid this, we developed a custom preprocessor
  (\cmdline{schemop}) that lets the developer write a cleaner code,
  focusing on the control flow of each job without worrying about
  defining the various support variables and functions that make the
  program run smoothly.

  \cmdline{schemop} is written in Python and offers a list of directives%
  \footnote{See \autoref{app:schemop}} that can be used in the
  source file in place of \ScheMo{}'s C++ macros or more complex statements.
  
  The preprocessor also tracks all used jobs, tasks, functions and
  related variables, automating their definition and initialization.
  
  Lastly, \cmdline{schemop} allows the management of shared memory
  by defining \textit{critical sections}, blocks of code that need
  to access a particular resource. This way, the developer need not
  to worry about defining, locking and unlocking mutexes since the
  preprocessor automatically generates the right code to do that.
  
  It is however worthy of note that only statically defined jobs and
  tasks can take advantage of the full capabilities of this tool,
  since it can not reliably predict runtime behaviour. \ScheMo{} allows
  dynamically created jobs and tasks, but mixing
  \code{@}-directives and normal macros is an operation that
  necessitates a clear understanding of the library and the
  preprocessor.
  
  Once the source file is given to \cmdline{schemop}, the latter
  converts it to a standard C++ source file, ready to be compiled
  normally with \cmdline{g++} or any other C++ compiler.
  
  \textit{For an example of \ScheMo{}'s features, see \autoref{app:code}.}
  
\subsubsection{\ScheMo{} Profiler}
\label{sssec:schemoprofile}
  A deeper understanding of the structure of a \ScheMo{} program and the
  way it is executed can help in its design and optimization.

  Because of this, we created a tool (called \ScheMo{} Profiler) that can
  analyse the program both by looking at the source code and at
  run-time\footnote{The current C++ implementation of the profiler uses
  the \code{ctime} and \code{cstdio} libraries, not available in all
  platforms. Some useful information can still be gathered by using it
  on the computer used to develop the software instead of the real
  device.}. The result is a \code{.profile} file, a text file that is
  both human-readable and easy to parse automatically.

  \ScheMo{} Profiler isn't made of standalone programs but it is
  a set of optional features built-in in both the \ScheMo{} Preprocessor
  and the \ScheMo{} C++ Library.
  
  \beforelist When used in conjunction with the \ScheMo{} Preprocessor, it:
  \begin{itemize}
    \item counts the number of jobs, functions, tasks and mutexes used
    \item assigns a name and ID to each job, function and task for
      additional analysis
    \item builds a flow graph of each job and function
  \end{itemize}
  \afterlist*
  \beforelist* If the profiler features have been enabled during compilation, when
  the resulting program is run it records in a \code{.profile} file
  (either an already existing one or a new one) data about how
  the scheduler executed the various parts of the program.
  Each time the scheduler passes control to a different job, in fact, it
  adds to the file:
  \begin{itemize}
    \item the ID of the currently running job
    \item the ID of the task the job is currently on
    \item a timestamp\footnote{\label{clock_note}Milliseconds elapsed
    since the program was launched.} of when control is passed
    \textit{to} the job
    \item a timestamp\footref{clock_note} of when control is passed
    \textit{away from} the job
  \end{itemize}
  \afterlist
  \beforelist Using this information it is possible to:
  \begin{itemize}
    \item reveal superflous pieces of code that are never accessed
    \item decide whether or not it is convenient to merge or split
    certain tasks based on their running time
    \item identify bottlenecks in the program execution
    \item recognise errors in the structure of the program
    \item etc\ldots
  \end{itemize}
  \afterlist
  As of yet, no memory-related profiling has been implemented.

  We also created a tool called \ScheMoTeX{} (and in particular the
  Python script \cmdline{smp2tikz}\footnote{PGF/Ti\textit{k}Z is a
  \LaTeX{} package used in the output files to visualize the graphs})
  that converts \ScheMo{} Profiler's \code{.profile} files in \LaTeX{}
  \code{.tex} files, making it possible to visualize the whole program
  and more intuitively analyse its structure. \autoref{fig:schemoflow}
  and the diagrams in \autoref{app:code} are examples of
  \ScheMoTeX{}'s output.
